%---------------------------------------------------------------------------------------------------------------------------------------------------------------------------------------------------------------------------------------------------------------------------------------------------------------------------------------------------------------------------------------------------
% Welcome to the notebook template. When every you see this symbol, %, it means that there is a comment and that it will not show up on the finished page. The notebook as shown is written in the coding language, LaTex; this language is an official notebook language for engineers, which is the main reason why we are using it. You can make a copy of this document if you like, but be sure to change everything that the comments tell you to change. This guide is made so that you (the person writing the notebook) doesn't have to know LaTex. But to make the notebook into the pdf that it needs to be, please download Texmaker (http://www.xm1math.net/texmaker/index.html). It allows one to code in LaTex and transfer a LaTex document to PDF form. Please also post both forms of your notebook in the NOTEBOOK folder on Google Drive. Remember to name it by date (e.g. september14.pdf). You can delete this comment section if you want. (everything in between the big long lines.) Good luck. Have fun.
% - Griffin Dugan.
%---------------------------------------------------------------------------------------------------------------------------------------------------------------------------------------------------------------------------------------------------------------------------------------------------------------------------------------------------------------------------------------------------


\documentclass[12pt]{article}
\usepackage{caption}
\usepackage{float}
\usepackage{graphicx}
\usepackage{fancyhdr}
\pagestyle{fancy}
\pagenumbering{Roman}
\renewcommand{\headrulewidth}{1pt}
\renewcommand{\footrulewidth}{1pt}
\setlength{\headheight}{25pt}
\rhead{\textbf{Lightning Coalition Robotics}}
\cfoot{}
\rfoot{\thepage}
\begin{document}

% Add date (e.g. September 14, 2018) and then your name/all authors.
Date - Author

\section{Our Plan:} % Pretty self explainatory... In this section explain the team's plan.
\begin{itemize}
% After \item, add what you want for the bullet point. (\item adds a new bullet point when you run out.)
	\item Item 1
\end{itemize}

% Now add a paragraph explaining your plan. You should reference the bullet points above.
Lorem ipsum dolor sit amet, consectetur adipiscing elit. Mauris id lacus et justo finibus mattis id et ante. Quisque a cursus enim. Donec sit amet fermentum sapien, eu scelerisque massa. Aliquam quis tempus turpis, sit amet faucibus dui. Proin at sem diam. Donec justo diam, accumsan a consequat eu, gravida in ligula. Proin interdum interdum egestas.

\section{What We Got Done:} % Just like the above, explain what the team got done during practice.
\begin{itemize}
% After \item, add what you want for the bullet point. (\item adds a new bullet point when you run out.)
	\item Item 2
\end{itemize}

% Now add a paragraph explaining what you got done and the reasons behind it. You should reference the bullet points above.
Integer hendrerit tempor libero vitae ornare. Sed et vestibulum nisl, non eleifend purus. Sed vel felis elit. Nulla auctor diam sem, vitae ornare neque fringilla at. Etiam interdum elit convallis nisl dapibus sagittis. Pellentesque facilisis neque ut erat pretium, non tempor turpis placerat. Quisque sit amet elit elit.

\section{What We Didn't Get Done:} % Explain what the team didn't get done during practice.
\begin{itemize}
% After \item, add what you want for the bullet point. (\item adds a new bullet point when you run out.)
	\item Item 3
\end{itemize}

% Now add a paragraph explaining what you didn't get done and the reasons behind it. You should reference the bullet points above.
Donec in mi vitae est blandit tempor eget ut dui. Nunc tincidunt lacus eget sem finibus pulvinar at ut neque. Maecenas vitae faucibus lorem. Phasellus condimentum faucibus dolor. Donec cursus luctus aliquam. Proin non arcu risus. Donec iaculis maximus porttitor.

\section{Next Practice:}
\begin{itemize}
% After \item, add what you want for the bullet point. (\item adds a new bullet point when you run out.)
	\item Item 4
\end{itemize}

% Now add a paragraph explaining what the plan for next practice should be. You should reference the bullet points above.
The next practice is... (e.g. Friday, September 14th, 2018). % And then continue the paragraph.
Pellentesque congue ac lacus ac aliquam. Mauris in metus malesuada, tincidunt quam vitae, pulvinar ligula. Nam arcu sem, faucibus eu lacinia vel, ultrices nec nibh. Donec vitae nibh dapibus, aliquet augue tristique, ullamcorper nibh. Cras in elementum sapien, nec malesuada felis. Curabitur vestibulum feugiat turpis.

\end{document}
